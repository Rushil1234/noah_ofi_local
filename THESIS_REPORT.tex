\documentclass[11pt, a4paper]{article}
\usepackage[utf8]{inputenc}
\usepackage[T1]{fontenc}
\usepackage{geometry}
\geometry{margin=1in}
\usepackage{graphicx}
\usepackage{booktabs}
\usepackage{hyperref}
\usepackage{listings}
\usepackage{xcolor}
\usepackage{float}
\usepackage{amsmath}
\usepackage{caption}
\usepackage{setspace}
\onehalfspacing

% Colors for code listings
\definecolor{codegreen}{rgb}{0,0.6,0}
\definecolor{codegray}{rgb}{0.5,0.5,0.5}
\definecolor{codepurple}{rgb}{0.58,0,0.82}
\definecolor{backcolour}{rgb}{0.95,0.95,0.92}

\lstdefinestyle{mystyle}{
    backgroundcolor=\color{backcolour},   
    commentstyle=\color{codegreen},
    keywordstyle=\color{magenta},
    numberstyle=\tiny\color{codegray},
    stringstyle=\color{codepurple},
    basicstyle=\ttfamily\footnotesize,
    breakatwhitespace=false,         
    breaklines=true,                 
    captionpos=b,                    
    keepspaces=true,                 
    numbers=left,                    
    numbersep=5pt,                  
    showspaces=false,                
    showstringspaces=false,
    showtabs=false,                  
    tabsize=2
}

\lstset{style=mystyle}

\title{\textbf{Order Flow Imbalance (OFI) Prediction Analysis} \\ \large Comprehensive Technical Report}
\author{Rushil Kakkad}
\date{January 6, 2026}

\begin{document}

\maketitle
\begin{abstract}
This study investigates the predictive structure of Order Flow Imbalance (OFI) for high-frequency trading applications. Using millisecond-level NBBO data for Apple Inc. (AAPL) and Microsoft Corp. (MSFT) throughout 2023, we construct a comprehensive dataset of 5-minute OFI bars. We demonstrate that OFI exhibits strong short-memory persistence (lag-1 autocorrelation $\approx 0.56$) but limited long-range dependence. Predictive modeling using XGBoost achieves a directional accuracy of 79\%, significantly outperforming deep learning approaches (LSTM). Furthermore, we document significant cross-asset Granger causality, where MSFT order flow provides incremental predictive information for AAPL. Importantly, we distinguish between informational predictability (sign accuracy) and economic alpha, noting that transaction costs likely constrain direct tradability.
\end{abstract}

\tableofcontents
\newpage

\section{Introduction}

Understanding the dynamics of order flow imbalance (OFI) is central to market microstructure, as order flow represents the primary mechanism through which information and liquidity demands are transmitted into prices. While prior literature has documented strong contemporaneous relationships between order flow and returns, the extent to which order flow exhibits predictable structure across time and across assets remains an open question.

In modern electronic markets, the limit order book (LOB) serves as the nexus of supply and demand. OFI, defined as the net buy vs. sell pressure at the best bid and ask quotes, is widely considered the most immediate driver of valid price changes. However, the academic consensus has often treated order flow as a martingale difference sequence---unpredictable in the efficient market limit. This thesis challenges that assumption by explicitly testing for temporal structure and forecasting power in aggregated OFI series.

The economic significance of OFI predictability extends beyond simple arbitrage. If order flow is predictable, it implies that liquidity provision is not instantaneous and that supply-demand imbalances persist over human-perceptible timescales (minutes). This persistence can induce "toxicity" for market makers, who may be adversely selected by informed traders splitting large parent orders. Conversely, for institutional execution algorithms, predicting future OFI allows for optimal scheduling of trades to minimize impact.

We focus specifically on the 5-minute aggregation horizon. This choice is deliberate: it bridges the gap between high-frequency microstructure noise (tick level) and low-frequency asset pricing factors (daily level). At this frequency, valid signals must survive the rapid cancellation behavior of HFT algorithms, representing robust, persistent institutional demand.

Furthermore, this study introduces a cross-asset dimension by analyzing the interplay between AAPL and MSFT. These assets are heavily linked through index funds (S\&P 500, Nasdaq 100) and ETFs (XLK, QQQ). We hypothesize that liquidity shocks to one asset diffuse to the other, not instantaneously, but with measurable lag. Validating this hypothesis provides specific evidence for the "common factor" theory of liquidity.

\section{Literature Review}

\subsection{Order Flow and Price Formation}
The theoretical foundation of this work rests on the seminal contributions of Kyle (1985) and Glosten and Milgrom (1985), who established that informed trading reveals itself through order flow. Cont, Kukanov, and Stoikov (2014) formalized the definition of Order Flow Imbalance (OFI) essentially as the net flow of limit orders at the best quotes. They demonstrated a near-linear relationship between contemporaneous OFI and price impact, a finding robust across asset classes.

Hasbrouck (1991) utilized vector autoregression (VAR) to decompose price changes into trade-related and information-related components, arguing that "trades drive quotes." Our work extends this by asking whether the trades (or specifically, the LOB imbalances) drive \textit{future} imbalances, a second-order effect implying persistence in liquidity demand.

\subsection{Short-Horizon Predictability}
Empirical studies on high-frequency predictability often encounter the efficient market hypothesis in its strongest form. While returns are notoriously difficult to forecast, order flow is widely acknowledged to possess "long memory" (Bouchaud et al., 2004). This persistence is attributed to order splitting: large institutional parents are sliced into child orders executed over time, creating a localized correlation in sign.

Despite this, machine learning applications often fail to produce trading alpha. As demonstrated by Sirignano and Cont (2019) with the "Universal Price Formation" model, deep learning shines on raw LOB states. However, for aggregated metrics, the signal-to-noise ratio often favors parsimonious models. Our findings confirm that gradient boosting (XGBoost) outperforms complex sequence models (LSTM) when features are carefully engineered.

\subsection{Cross-Asset Microstructure}
The literature on cross-asset effects pivots on lead-lag relationships in \textit{liquidity}. Beta arbitrage and index arbitrage create mechanical links between assets. When an ETF is bought, Market Makers hedge by buying the constituents. Since execution is not simultaneous, order flow in highly liquid leaders (like MSFT) often precedes flow in peers. This "common factor" view suggests that systematic OFI drives trends while idiosyncratic OFI is noise.

\section{Data Extraction and Processing}

\subsection{Data Source}
\begin{itemize}
    \item \textbf{Database:} WRDS TAQ Millisecond (TAQM)
    \item \textbf{Tables:} \texttt{taqm\_2023.nbbom\_YYYYMMDD} (NBBO updates)
    \item \textbf{Symbols:} AAPL, MSFT
    \item \textbf{Period:} January 3, 2023 -- December 29, 2023
    \item \textbf{Trading Hours:} 09:30--16:00 (Regular trading only)
\end{itemize}

\subsection{Extraction Statistics}
\begin{table}[H]
    \centering
    \begin{tabular}{ll}
        \toprule
        \textbf{Metric} & \textbf{Value} \\
        \midrule
        Total parquet files & 500 \\
        Total 5-min bars & 38,998 (per symbol) \\
        Combined dataset & 77,996 rows \\
        \bottomrule
    \end{tabular}
    \caption{Data Summary}
\end{table}

\subsection{OFI Computation Logic}
Order Flow Imbalance is computed efficiently at the event level. We track changes in the Best Bid and Best Ask to infer the presence of limit orders initiated by buyers or sellers.

\textbf{Bid Contribution ($e_{bid}$):}
\begin{itemize}
    \item If $P^{bid}_t > P^{bid}_{t-1}$: Limit Buy (Add Liquidity) $\to +q^{bid}_t$
    \item If $P^{bid}_t < P^{bid}_{t-1}$: Market Sell (Consume Liquidity) $\to -q^{bid}_{t-1}$
    \item If $P^{bid}_t = P^{bid}_{t-1}$: Change in depth $\to +(q^{bid}_t - q^{bid}_{t-1})$
\end{itemize}

\textbf{ask Contribution ($e_{ask}$):} Analogous logic for the ask side.

\subsection{SQL Implementation}
To handle the massive scale of millisecond data efficiently, we utilize PostgreSQL window functions within the WRDS environment before downloading:

\begin{lstlisting}[language=SQL, caption=OFI Event Calculation]
WITH nbbo AS (
    SELECT 
        time_m,
        best_bid, best_bidsiz, best_ask, best_asksiz,
        LAG(best_bid) OVER (ORDER BY time_m) AS prev_bid,
        LAG(best_bidsiz) OVER (ORDER BY time_m) AS prev_bidsiz,
        -- ... [similar for ask]
        FLOOR(EXTRACT(EPOCH FROM time_m) / 300) AS bucket
    FROM taqm_2023.nbbom_YYYYMMDD
    WHERE sym_root = 'AAPL' AND time_m >= '09:30:00' AND time_m < '16:00:00'
),
ofi_events AS (
    SELECT bucket,
        CASE WHEN best_bid > prev_bid THEN best_bidsiz
             WHEN best_bid < prev_bid THEN -prev_bidsiz
             ELSE best_bidsiz - prev_bidsiz END AS e_bid,
        -- ... [similar for e_ask]
    FROM nbbo
)
SELECT bucket, SUM(e_bid + e_ask) AS ofi FROM ofi_events GROUP BY bucket
\end{lstlisting}

\section{Methodology}

\subsection{Feature Engineering}
We construct a feature set designed to capture both immediate momentum and potential mean reversion.

\begin{table}[H]
    \centering
    \begin{tabular}{ll}
        \toprule
        \textbf{Feature} & \textbf{Description} \\
        \midrule
        \texttt{ofi\_z} & Z-score normalized OFI (using training set stats) \\
        \texttt{lag1} & $OFI_{t-1}$ \\
        \texttt{lag2} & $OFI_{t-2}$ \\
        \texttt{lag3} & $OFI_{t-3}$ \\
        \texttt{lag6} & $OFI_{t-6}$ (30 min ago) \\
        \texttt{lag12} & $OFI_{t-12}$ (1 hour ago) \\
        \texttt{y\_next} & Target: $OFI_{t+1}$ \\
        \bottomrule
    \end{tabular}
    \caption{Features}
\end{table}

\subsection{Normalization Protocols (Leakage Prevention)}
A critical methodological step is ensuring no look-ahead bias enters the normalization. Standard z-scoring using the entire dataset's mean and standard deviation would constitute leakage. Instead, we calculate statistics \textit{only} on the Training split:

\begin{lstlisting}[language=Python, caption=Leak-free Normalization]
# Compute specific statistics from training period Only
train_mean = train_df['ofi'].mean()  
train_std = train_df['ofi'].std()

# Apply these fixed scalars to Train, Validation, and Test
df['ofi_z'] = (df['ofi'] - train_mean) / train_std
\end{lstlisting}

\subsection{Validation Strategy}
We employ a strict time-series split:
\begin{itemize}
    \item \textbf{Train:} Jan -- Sep 2023 (75\%)
    \item \textbf{Validation:} Oct -- Nov 2023 (17\%)
    \item \textbf{Test:} Dec 2023 (8\%)
\end{itemize}

\section{Empirical Results}

\subsection{Predictive Performance}
We compare XGBoost (Gradient Boosting), LSTM (Recurrent Neural Network), and a baseline AR(1) model.

\textbf{Note regarding "Accuracy":} Directional Accuracy refers to predicting the \textit{sign of future OFI}, not the sign of future returns. While linked, they are distinct; high OFI accuracy implies predictable order flow pressure, which is a precursor to price movement but not a guarantee of profitable trading alpha after costs.

\begin{table}[H]
    \centering
    \begin{tabular}{lllll}
        \toprule
        \textbf{Model} & \textbf{MSE} & \textbf{MAE} & \textbf{Correlation} & \textbf{Dir. Accuracy} \\
        \midrule
        \textbf{XGBoost} & \textbf{0.499} & \textbf{0.339} & \textbf{70.65\%} & \textbf{79.16\%} \\
        LSTM & 0.975 & 0.603 & 15.26\% & 59.05\% \\
        AR(1) & 0.969 & N/A & 16.15\% & 59.11\% \\
        \bottomrule
    \end{tabular}
    \caption{Test Set Performance (Dec 2023)}
\end{table}

The XGBoost model massively outperforms the baselines. The LSTM's poor performance suggests that the signal does not rely on complex, long-range dependencies that require memory cells, but rather on immediate, locally non-linear thresholds that trees capture well.

\subsection{Ablation Studies}
To understand the source of predictability, we systematically remove features:

\begin{table}[H]
    \centering
    \begin{tabular}{lllll}
        \toprule
        \textbf{Configuration} & \textbf{Features} & \textbf{MSE} & \textbf{Corr} & \textbf{Dir.Acc} \\
        \midrule
        \textbf{Full model} & ofi\_z + lags & \textbf{0.499} & \textbf{70.65\%} & \textbf{79.16\%} \\
        Only ofi\_z & Current bar only & 0.659 & 58.19\% & 78.95\% \\
        Lags only & lags 1-12 & 0.963 & 17.66\% & 58.96\% \\
        Only lag1 & Single lag & 0.964 & 17.70\% & 59.27\% \\
        \bottomrule
    \end{tabular}
    \caption{Feature Ablation Results}
\end{table}

\textbf{Key Insight:} The current bar's OFI is the dominant predictor ($R \approx 58\%$). However, including history (lags) boosts correlation and MSE significantly, even if directional accuracy saturates. This implies the \textit{magnitude} of the next imbalance is conditioned on the path, even if the \textit{direction} is largely persistent.

\subsection{Autocorrelation Structure}
The predictability is driven by strong short-memory persistence:

\begin{table}[H]
    \centering
    \begin{tabular}{llll}
        \toprule
        \textbf{Symbol} & \textbf{Lag-1 ACF} & \textbf{Lag-6 ACF} & \textbf{Lag-12 ACF} \\
        \midrule
        AAPL & 0.5653 & 0.0195 & -0.0079 \\
        MSFT & 0.5697 & 0.0543 & 0.0165 \\
        \bottomrule
    \end{tabular}
    \caption{Autocorrelation Function}
\end{table}

\subsection{Cross-Asset Analysis}
We investigate if order flow in one asset predicts the other.

\textbf{Contemporaneous Correlation:} $\rho(OFI_{AAPL}, OFI_{MSFT}) = 35.33\%$

\textbf{Granger Causality Tests:}
\begin{table}[H]
    \centering
    \begin{tabular}{llll}
        \toprule
        \textbf{Direction} & \textbf{F-statistic} & \textbf{p-value} & \textbf{Result} \\
        \midrule
        AAPL $\to$ MSFT & 2.74 & 0.0272 & Significant ($<0.05$) \\
        \textbf{MSFT $\to$ AAPL} & \textbf{5.46} & \textbf{0.0002} & \textbf{Highly Significant} \\
        \bottomrule
    \end{tabular}
    \caption{Granger Causality Results}
\end{table}

MSFT OFI provides strong incremental information for predicting AAPL OFI. This asymmetry may reflect MSFT's higher weight in certain indices or specific ETF flows, acting as a leading indicator for the sector.

\section{Discussion and Limitations}

\subsection{Interpretation of Results}
Our findings confirm that 5-minute aggregated OFI is not a random walk. It follows an AR(1)-like process with significant persistence. The superiority of XGBoost over LSTM and AR(1) suggests that while the primary component is linear persistence, there are non-linear regime effects (captured by tree splits) that improve magnitude forecasting.

\subsection{Execution vs. Informational Alpha}
A directional accuracy of 79\% is statistically powerful but economically nuanced. This predictability applies to the \textit{imbalance itself}. To monetize this via proprietary trading, one must cross the bid-ask spread. Given the tight spreads of AAPL/MSFT, the predicted OFI magnitude must imply a price move larger than the effective spread + fees. Thus, this signal is likely most valuable for \textit{execution algorithms} (e.g., "wait 5 minutes to buy because sell pressure is persisting") rather than standalone alpha.

\subsection{Limitations}
\begin{itemize}
    \item \textbf{Two Assets:} Results may not generalize to small-cap stocks with sparse books.
    \item \textbf{Single Year:} 2023 was a specific volatility regime; 2020 or 2008 might show different dynamics.
    \item \textbf{Level 1 Data:} We ignore deep book imbalances, which might contain additional signal.
\end{itemize}

\section{Conclusion}
This thesis successfully establishes that Order Flow Imbalance in major US equities is a predictable, persistent process at the 5-minute horizon. By rigorously processing WRDS TAQM data and employing leak-free validation, we show that XGBoost models can forecast the sign of future order imbalances with $\approx 79\%$ accuracy. The discovery of a strong MSFT $\to$ AAPL lead-lag relationship contributes empirical evidence to the theory of common factor liquidity. Future work should focus on linking this OFI predictability directly to realized price returns to quantify the actionable economic value.

\end{document}
